% knitr mixed latex en R tot pdf
\documentclass[10pt,a4paper,titlepage]{report}\usepackage{graphicx, color}
%% maxwidth is the original width if it is less than linewidth
%% otherwise use linewidth (to make sure the graphics do not exceed the margin)
\makeatletter
\def\maxwidth{ %
  \ifdim\Gin@nat@width>\linewidth
    \linewidth
  \else
    \Gin@nat@width
  \fi
}
\makeatother

\IfFileExists{upquote.sty}{\usepackage{upquote}}{}
\definecolor{fgcolor}{rgb}{0.2, 0.2, 0.2}
\newcommand{\hlnumber}[1]{\textcolor[rgb]{0,0,0}{#1}}%
\newcommand{\hlfunctioncall}[1]{\textcolor[rgb]{0.501960784313725,0,0.329411764705882}{\textbf{#1}}}%
\newcommand{\hlstring}[1]{\textcolor[rgb]{0.6,0.6,1}{#1}}%
\newcommand{\hlkeyword}[1]{\textcolor[rgb]{0,0,0}{\textbf{#1}}}%
\newcommand{\hlargument}[1]{\textcolor[rgb]{0.690196078431373,0.250980392156863,0.0196078431372549}{#1}}%
\newcommand{\hlcomment}[1]{\textcolor[rgb]{0.180392156862745,0.6,0.341176470588235}{#1}}%
\newcommand{\hlroxygencomment}[1]{\textcolor[rgb]{0.43921568627451,0.47843137254902,0.701960784313725}{#1}}%
\newcommand{\hlformalargs}[1]{\textcolor[rgb]{0.690196078431373,0.250980392156863,0.0196078431372549}{#1}}%
\newcommand{\hleqformalargs}[1]{\textcolor[rgb]{0.690196078431373,0.250980392156863,0.0196078431372549}{#1}}%
\newcommand{\hlassignement}[1]{\textcolor[rgb]{0,0,0}{\textbf{#1}}}%
\newcommand{\hlpackage}[1]{\textcolor[rgb]{0.588235294117647,0.709803921568627,0.145098039215686}{#1}}%
\newcommand{\hlslot}[1]{\textit{#1}}%
\newcommand{\hlsymbol}[1]{\textcolor[rgb]{0,0,0}{#1}}%
\newcommand{\hlprompt}[1]{\textcolor[rgb]{0.2,0.2,0.2}{#1}}%

\usepackage{framed}
\makeatletter
\newenvironment{kframe}{%
 \def\at@end@of@kframe{}%
 \ifinner\ifhmode%
  \def\at@end@of@kframe{\end{minipage}}%
  \begin{minipage}{\columnwidth}%
 \fi\fi%
 \def\FrameCommand##1{\hskip\@totalleftmargin \hskip-\fboxsep
 \colorbox{shadecolor}{##1}\hskip-\fboxsep
     % There is no \\@totalrightmargin, so:
     \hskip-\linewidth \hskip-\@totalleftmargin \hskip\columnwidth}%
 \MakeFramed {\advance\hsize-\width
   \@totalleftmargin\z@ \linewidth\hsize
   \@setminipage}}%
 {\par\unskip\endMakeFramed%
 \at@end@of@kframe}
\makeatother

\definecolor{shadecolor}{rgb}{.97, .97, .97}
\definecolor{messagecolor}{rgb}{0, 0, 0}
\definecolor{warningcolor}{rgb}{1, 0, 1}
\definecolor{errorcolor}{rgb}{1, 0, 0}
\newenvironment{knitrout}{}{} % an empty environment to be redefined in TeX

\usepackage{alltt}

\usepackage[latin1]{inputenc}
\usepackage[english]{babel}
\usepackage{amsmath}
\usepackage{amsfonts}
\usepackage{amssymb}
\usepackage{graphicx}
% bepaald marges van papier
\usepackage[left=2cm,right=2cm,top=2cm,bottom=2cm]{geometry}
% voor plaatsten van figure
\usepackage{float}
% voor tabellen
\usepackage{longtable}

% header and footer definition
\usepackage{fancyhdr}
\pagestyle{fancy}

\lhead{\includegraphics[height=0.8cm]{dikw-logo.png}}
\chead{}
\rhead{\bfseries Data Analysis Report}
\lfoot{}
\cfoot{Data $>$ Information $>$ Knowledge $>$ Wisdom}
\rfoot{\thepage}
\renewcommand{\headrulewidth}{0.4pt}
\renewcommand{\footrulewidth}{0.4pt}

\author{Hugo Koopmans}
\title{Data Analysis Report}
\begin{document}

\maketitle
\tableofcontents
\newpage
\section{Introduction}
This data analysis report is generated using R, R-studio and knitr to sweave R code and Latex into pdf format. We have the option to include all R code that is used to generate the plots and calculations. Default this feauture is dissabled.\\
The data analysis step is the first step an a datamining analysis.
\section{Dataset Basic Artifactes}

\subsection{Basic dataset information}
Basic information on dataset:\\
\begin{knitrout}
\definecolor{shadecolor}{rgb}{0.969, 0.969, 0.969}\color{fgcolor}\begin{kframe}


{\ttfamily\noindent\textcolor{warningcolor}{\#\# Warning: cannot open file 'data-analysis-template.R': No such file or directory}}

{\ttfamily\noindent\bfseries\textcolor{errorcolor}{\#\# Error: cannot open the connection}}\end{kframe}
\end{knitrout}





\\
Read data from file : 

{\ttfamily\noindent\bfseries\textcolor{errorcolor}{\\Error in eval(expr, envir, enclos) : object 'filename' not found}}.\\
The dataset has 

{\ttfamily\noindent\bfseries\textcolor{errorcolor}{\\Error in eval(expr, envir, enclos) : object 'colums' not found}} variables and 

{\ttfamily\noindent\bfseries\textcolor{errorcolor}{\\Error in eval(expr, envir, enclos) : object 'rows' not found}} rows.

The case identifyer is 

{\ttfamily\noindent\bfseries\textcolor{errorcolor}{\\Error in eval(expr, envir, enclos) : object 'original\_case\_id' not found}} this is unique for all cases.

\subsection{Excluded variables}
From the varables provided the folowing list will be excluded in this anlysis: 

{\ttfamily\noindent\bfseries\textcolor{errorcolor}{\\Error in eval(expr, envir, enclos) : object 'exclude\_var\_names' not found}}

\subsection{Variabele types}


The following variabele are present in the dataset:\\


{\ttfamily\noindent\bfseries\textcolor{errorcolor}{\\Error in eval(expr, envir, enclos) : object 'var\_names' not found}} 
\\
Sometimes categoric variables are present as coded numbers. These should be treated as factors.
In this dataset the following variables will be used as factors(categoric): 

{\ttfamily\noindent\bfseries\textcolor{errorcolor}{\\Error in eval(expr, envir, enclos) : \\  object 'treat\_as\_categorical' not found}}
\\
We have 

{\ttfamily\noindent\bfseries\textcolor{errorcolor}{\\Error in eval(expr, envir, enclos) : object 'num\_vars' not found}} numeric variables and 

{\ttfamily\noindent\bfseries\textcolor{errorcolor}{\\Error in eval(expr, envir, enclos) : object 'cat\_vars' not found}} categorical variables (or factors in R).
\newpage
\section{Numeric variables}
Here we analyse all numeric variables. We start with an overview on basic statistics per variable. We check for missing values. We do a histogram plot to show the distribution for this variable. And we test for outliers.

\subsection{Overview}
In the table below we report the number of observations (n), the smallest observation (min),  the first quantile (q1), the media ,  the mean, last quantile, the largest observation (max), and the nber of missing values (na).\\




 





{\ttfamily\noindent\bfseries\textcolor{errorcolor}{\\Error in paste(out, collapse = "\textbackslash{}n") : object 'out' not found}}

% analyse categorical variabels
\newpage
\section{Categorical variables}
Here we analyse all categorical variables. We first check the number of different levels in each category(or factor). Then we do a bar plot to show the distribution for each variable.
\\
Overview\\
In the following table we will see each variable printed with it's unique levels. Beside each level a count is made and a precentage calculated. In the last colum we find a culumative count summing the total up to 100\%. 


We see that the number of levels can be quite big, for reporting we will omit all variables with more then  

{\ttfamily\noindent\bfseries\textcolor{errorcolor}{\\Error in eval(expr, envir, enclos) : object 'max\_levels' not found}} levels. These will not be reported in the subsections below.



Variables with to many levels to report are : 

{\ttfamily\noindent\bfseries\textcolor{errorcolor}{\\Error in eval(expr, envir, enclos) : \\  object 'cat\_var\_names\_not\_reported' not found}}






{\ttfamily\noindent\bfseries\textcolor{errorcolor}{\\Error in paste(out, collapse = "\textbackslash{}n") : object 'out' not found}}

\section{Behavioural Analysis}
The next step is behavioural analysis.
The current dataset is now saved.\\



Dataset saved as : 

{\ttfamily\noindent\bfseries\textcolor{errorcolor}{\\Error in eval(expr, envir, enclos) : object 'datasetName' not found}}

\end{document}
