% knitr mixed latex en R tot pdf
\documentclass[10pt,a4paper,titlepage]{report}

\usepackage[latin1]{inputenc}
\usepackage[english]{babel}
\usepackage{amsmath}
\usepackage{amsfonts}
\usepackage{amssymb}
\usepackage{graphicx}
% bepaald marges van papier
\usepackage[left=2cm,right=2cm,top=2cm,bottom=2cm]{geometry}
% voor plaatsten van figure
\usepackage{float}
% voor tabellen
\usepackage{longtable}

% header and footer definition
\usepackage{fancyhdr}
\pagestyle{fancy}

\lhead{\includegraphics[height=0.8cm]{dikw-logo.png}}
\chead{}
\rhead{\bfseries Data Analysis Report}
\lfoot{}
\cfoot{Data $>$ Information $>$ Knowledge $>$ Wisdom}
\rfoot{\thepage}
\renewcommand{\headrulewidth}{0.4pt}
\renewcommand{\footrulewidth}{0.4pt}

\author{Hugo Koopmans}
\title{Data Analysis Report}
\begin{document}

\maketitle
\tableofcontents
\newpage
\section{Introduction}
This data analysis report is generated using R, R-studio and knitr to sweave R code and Latex into pdf format. We have the option to include all R code that is used to generate the plots and calculations. Default this feauture is dissabled.\\
The data analysis step is the first step an a datamining analysis.
\section{Dataset Basic Artifactes}

\subsection{Basic dataset information}
Basic information on dataset:\\




\begin{knitrout}
\definecolor{shadecolor}{rgb}{0.969, 0.969, 0.969}\color{fgcolor}\begin{kframe}
\begin{verbatim}
## [1] "/Users/toni/Dropbox/_Scratch/github/dataMineR/1-data-understanding"
## [1] "/Users/toni/Dropbox/_Scratch/github/dataMineR"
\end{verbatim}
\end{kframe}
\end{knitrout}

\\
Read data from file : data/data-simple-example.tab.\\
The dataset has 1 variables and 5000 rows.

The case identifyer is registrnr this is unique for all cases.

\subsection{Excluded variables}
From the varables provided the folowing list will be excluded in this anlysis: caseID, registrnr, X2011tmoktstornaant, X2010stornoaantal

\subsection{Variabele types}


The following variabele are present in the dataset:\\
 
\\
Sometimes categoric variables are present as coded numbers. These should be treated as factors.
In this dataset the following variables will be used as factors(categoric): catHHINKOMEN, catHHSOCIALE, catHHOPLEIDI, catHHLEVENSF, catHHGEOTYPE, catHHTYPEWO, catHHEIGENDO, catHHWOZWAA, catBELEGGERS, catLENERS, catSPAARDERS, catSWITCHGEVO, catMERKENTROU
\\
We have 0 numeric variables and 5013 categorical variables (or factors in R).
\newpage
\section{Numeric variables}
Here we analyse all numeric variables. We start with an overview on basic statistics per variable. We check for missing values. We do a histogram plot to show the distribution for this variable. And we test for outliers.

\subsection{Overview}
In the table below we report the number of observations (n), the smallest observation (min),  the first quantile (q1), the media ,  the mean, last quantile, the largest observation (max), and the nber of missing values (na).\\

\begin{knitrout}
\definecolor{shadecolor}{rgb}{0.969, 0.969, 0.969}\color{fgcolor}\begin{kframe}


{\ttfamily\noindent\bfseries\color{errorcolor}{\#\# Error: there is no package called 'reporttools'}}

{\ttfamily\noindent\bfseries\color{errorcolor}{\#\# Error: could not find function "tableContinuous"}}\end{kframe}
\end{knitrout}


 
% <<run-numeric, eval=TRUE, include=FALSE>>=
\begin{knitrout}
\definecolor{shadecolor}{rgb}{0.969, 0.969, 0.969}\color{fgcolor}\begin{kframe}


{\ttfamily\noindent\color{warningcolor}{\#\# Warning: cannot open file '1-data-understanding/CopyOfda-numeric-template.Rnw': No such file or directory}}

{\ttfamily\noindent\bfseries\color{errorcolor}{\#\# Error: cannot open the connection}}\end{kframe}
\end{knitrout}




% analyse categorical variabels
\newpage
\section{Categorical variables}
Here we analyse all categorical variables. We first check the number of different levels in each category(or factor). Then we do a bar plot to show the distribution for each variable.
\\
Overview\\
In the following table we will see each variable printed with it's unique levels. Beside each level a count is made and a precentage calculated. In the last colum we find a culumative count summing the total up to 100\%. 
\begin{knitrout}
\definecolor{shadecolor}{rgb}{0.969, 0.969, 0.969}\color{fgcolor}\begin{kframe}


{\ttfamily\noindent\bfseries\color{errorcolor}{\#\# Error: there is no package called 'reporttools'}}

{\ttfamily\noindent\bfseries\color{errorcolor}{\#\# Error: incorrect number of dimensions}}

{\ttfamily\noindent\bfseries\color{errorcolor}{\#\# Error: incorrect number of dimensions}}

{\ttfamily\noindent\bfseries\color{errorcolor}{\#\# Error: object 'cat\_dat\_limited\_levels' not found}}

{\ttfamily\noindent\bfseries\color{errorcolor}{\#\# Error: object 'cat\_dat\_not\_reported' not found}}

{\ttfamily\noindent\bfseries\color{errorcolor}{\#\# Error: object 'cat\_var\_names\_lim' not found}}\end{kframe}
\end{knitrout}

We see that the number of levels can be quite big, for reporting we will omit all variables with more then  25 levels. These will not be reported in the subsections below.
\begin{knitrout}
\definecolor{shadecolor}{rgb}{0.969, 0.969, 0.969}\color{fgcolor}\begin{kframe}


{\ttfamily\noindent\bfseries\color{errorcolor}{\#\# Error: there is no package called 'reporttools'}}

{\ttfamily\noindent\bfseries\color{errorcolor}{\#\# Error: 'x' must be an array of at least two dimensions}}

{\ttfamily\noindent\bfseries\color{errorcolor}{\#\# Error: object 'cat\_num\_missing' not found}}

{\ttfamily\noindent\bfseries\color{errorcolor}{\#\# Error: could not find function "xtable"}}

{\ttfamily\noindent\bfseries\color{errorcolor}{\#\# Error: object 'xt' not found}}

{\ttfamily\noindent\bfseries\color{errorcolor}{\#\# Error: object 'xt' not found}}

{\ttfamily\noindent\bfseries\color{errorcolor}{\#\# Error: object 'xt' not found}}\end{kframe}
\end{knitrout}


Variables with to many levels to report are : 

{\ttfamily\noindent\bfseries\color{errorcolor}{\\Error in eval(expr, envir, enclos) : \\\ \ object 'cat\_var\_names\_not\_reported' not found}}





